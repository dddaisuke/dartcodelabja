\section{Step 1: Set up your environment}

Dart offers better productivity through powerful and helpful tools. At the center of the toolchain is the Dart Editor, a lightweight text editor that understands how to analyze, run, and debug Dart apps. The editor works with the Dart SDK and Dartium (a build of Chromium with the Dart VM) to give you an integrated experience.

\subsection{Objectives}

1. Install Dart Editor
2. Send feedback to the editor team
3. Run the sample clock Dart app
4. Learn about Dartium

\subsection{Walkthrough}

\subsubsection{Install the Dart Editor}

To get your environment set up, plug in the provided USB. Open the USB drive and find the \url{editor/} directory inside. Copy over the correct Dart Editor version for your OS/bit combination directory to your machine, and unzip it.

[Image]

Open up the newly unzipped \url{dart/} directory, and double click the executable:

\begin{itemize}
\item DartEditor.app (Mac)
\item DartEditor (Linux)
\item DartEditor.exe (Windows)
\end{itemize}

[Image]

When Dart Editor first opens, the Welcome window appears.

[Image]

\subsubsection{Use the Send Feedback button}

The Dart Editor team really appreciates feedback. The easiest way to let them know your thoughts is to use the Send Feedback button in the upper right of the editor’s toolbar.

Locate the Send Feedback button and click on it.

[Image]

The Dart Editor Feedback dialog allows you to share bugs and requests directly with the editor team, as well as the larger Dart team. Feel free to send us any and all comments, especially during this codelab. We’ll turn your suggestions into bug reports and feature requests3 as appropriate.

[Image]

\subsubsection{Run the Clock sample}

(If the feedback window is still open, close it now.)

Time to run a Dart app!

Click on the Clock sample from the Welcome window, which copies the Clock sample into Dart Editor and sets up a new project for you.

[Image]

Tip: Can’t find the Welcome view? You can go to Tools, Welcome Page to display it again.

[Image]

The Files view shows all of the files in the Clock sample, including all Dart files, the HTML file that hosts the app, as well as all of the images and CSS files. clock.dart is a Dart file that defines the "clock" library, and includes the main() method for the sample. The clock.dart file is automatically opened in the editor.

[Image]

Ensure the clock.dart file is selected and highlighted. Click the Run button in Dart Editor, which launches the application by loading Dartium and pointing it to the clock.html file.

[Image]

Here is the Clock sample app running inside Dartium. Congratulations, you are running your first Dart app!

[Image]

\subsubsection{Dartium}

Dartium is Chromium with an embedded Dart virtual machine (VM). Even though Dart compiles to JavaScript, you can speed up the "edit, reload" development cycle by running Dart apps directly in the browser.

To verify that Clock is running in Dartium, right-click in the browser and select Inspect Element.

[Image]

The Elements tab should be selected by default, and clock.dart should be listed as the script that is being run.

[Image]

\subsubsection{Debug with the Dart Editor}

With Dart running directly in Dartium, the editor has debugging support for Dart applications. Set a breakpoint by double clicking in the left hand gutter in Dart Editor on first call to setDigits() inside the updateTime() method in clock.dart, line 66. With the breakpoint set (you will see a little blue dot in the gutter), click the Run button again.

[Image]

Notice how the program stop, and the Debugger view opens in the Dart Editor. To continue the program without leaving the debugger, click the Resume button (the green arrow in the Debugger view). The updateTime() method is called every 1000ms, therefore the breakpoint will be hit again within a second.

[Image]

The Debugger view on the right hand side allows you to see which processes are running and what values are in scope at the breakpoint. Hover over the now field, the value will be displayed in a tooltip.

[Image]

To terminate the debugger, click on the red square in the Debugger view. This will also stop the application.

[Image]

\subsection{Advanced}

Load and launch the other samples.
In the Clock sample, try changing the ball velocity or the gravity. Start with lines 12 and 117 in balls.dart.
Set more breakpoints and inspect the values by opening up the variables.

[Image]
